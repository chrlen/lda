\documentclass[15pt,landscape,twopage]{article}
\usepackage{amsmath,amssymb,amsfonts,amsthm}
\usepackage[english]{babel}
\usepackage[utf8]{inputenc}
\usepackage{fancyhdr}
%\usepackage[a3paper]{geometry}
\usepackage{geometry}
\usepackage{tikz}
\usepackage{xcolor}
\usepackage{float}
\usepackage{bm}
\usepackage{listings}
\usepackage{xcolor}
\usepackage{algorithm2e}
\usepackage{tikz}
\usetikzlibrary{automata,positioning}

\lstset{basicstyle=\ttfamily,
 showstringspaces=false,
 commentstyle=\color{gray},
 keywordstyle=\color{blue},
 morekeywords={grep, head, tail, cut }
}

\newcommand{\codepath}{../}
\newcommand{\scriptpath}{../scripts}

\newcommand{\hr}{\begin{center} \line(1,0){450} \end{center}}
\newcommand{\R}{\mathbb R}
\newcommand{\B}{ \{0,1\} }
\newcommand{\tr}{^\mathsf{T}}
\newcommand{\F}{\mathcal{F}}
\newcommand{\norm}[1]{\left\lVert#1\right\rVert}
\DeclareMathOperator{\Id}{Id}
\DeclareMathOperator{\diag}{diag}
\DeclareMathOperator{\epi}{epi}
\DeclareMathOperator{\co}{co}
\DeclareMathOperator{\interior}{int}
\DeclareMathOperator{\Proj}{Pr}
\DeclareMathOperator*{\argmin}{argmin}
\DeclareMathOperator*{\argmax}{argmax}
\newcommand{\midd}{\mathrel{}\middle|\mathrel{}}


\newcommand{\generalPdfSymbol}{\bm{p}}
\newcommand{\gps}{\generalPdfSymbol}
\newcommand{\bernoulliPdfSymbol}{p_{bern}}
\newcommand{\binomialPdfSymbol}{p_{bin}}

\newcommand{\s}[1]{\sum_{#1}}
\newcommand{\am}[2]{\argmax\limits_{#1} #2}
\newcommand{\partition}[2]{\{ #1_1, #1_2, \dots, #1_{#2} \} }
\newcommand{\pd}[2]{#1 \left( #2 \right) }
\newcommand{\p}[1]{\pd{P}{#1}}
\newcommand{\cp}[3]{\pd{#1}{#2 \midd #3}}
\newcommand{\cpf}[3]{ \frac{\pd{#1}{#2,#3}} { \pd{#1}{#3} } }
\newcommand{\cpeq}[3]{\cp{#1}{#2}{#3} = \cpf{#1}{#2}{#3}}

\newcommand{\ltp}[3]{\s{#2_j \in \partition{#2}{n} } \cp{#1}{#3}{#2_j} \p{#2_j} }
\newcommand{\bysNom}[3]{\cp{#1}{#3}{#2_i} \pd{#1}{#2_i} }
\newcommand{\bys}[3]{ \frac{\bysNom{#1}{#2}{#3} }{ \ltp{#1}{#2}{#3} } }
\newcommand{\byseq}[3]{}


\newcommand{\bernoulli}[2]{ #1^{#2} \left(1 - #1\right)^{1 - #2}  }
\newcommand{\bern}{\bernoulli{\phi}{c}}

\newcommand{\likelihood}[1]{\prod\limits_{c \in C} #1 }


\geometry{top=20mm, left=20mm, right=10mm, bottom=15mm}

\pagestyle{fancy}
\lhead{Christian Lengert 153767}
\rhead{\today}
\chead{Graphical Models Lab : Latent Dirichlet Allocation (LDA)}
\rfoot{Page \thepage}

\begin{document}
\section{The model of Latent Dirichlet Allocation}
The model of Latent Dirichlet Allocation is a
\section{Imlementation}
\subsection{Preparation}
First we have to choose a text corpus to estimate the parameters of the LDA-model from. For this we use a dump from the simple english wikipedia \cite{simplewi84:online}, namely the version named 'All pages, current versions only'.

For the purpose of speeding up the development process we will perform our operations on an even smaller subset of only 5 articles, to avoid long loading times on every run. We split some articles of the downloaded file into a smaller file with the script in section \ref{extract}.

\subsection{Class : Dataset}
We concentrate all operations regarding the preprocessing of the data in class named Dataset, which can be examined in section  \ref{class:dataset}. Our goal here is to process all articles and end up with a matrix of wordcounts, each row representing an article and each column representing a word. Therefore we have to establish a common dictionary containing all occurring words from all

\subsection{Class : LDA}

\section{Applying LDA}
\newpage

\bibliography{refs}
\bibliographystyle{IEEEtran}

\newpage

\section{Appendix}
\subsection{Extract subset} \label{extract}
\lstinputlisting[language=bash]{\scriptpath/extractSmallSubset.bash}

\subsection{Class : Dataset} \label{class:dataset}
\lstinputlisting[language=python]{\codepath/dataset.py}

\subsection{Class : LDA} \label{class:lda}
\lstinputlisting[language=python]{\codepath/inference.py}



\end{document}
